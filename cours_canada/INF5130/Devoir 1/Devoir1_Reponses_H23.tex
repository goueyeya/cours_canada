  \documentclass[11pt]{article}
 \setlength\headheight{15pt} 
 \usepackage[left=2cm,top=3cm,right=2cm,bottom=3cm]{geometry}
  \usepackage[utf8]{inputenc}
    \usepackage{setspace}
    \usepackage{tabto}
  \usepackage{enumitem}
  \usepackage{fancyhdr}
  \pagestyle{fancy}
\usepackage{amsfonts,amsmath,amssymb}
\usepackage{graphicx}


\begin{document}

\thispagestyle{empty}

\begin{center}
\textbf{\Large Université du Qu\'ebec \`a Montr\'eal}

\bigskip
\bigskip

\textbf{\LARGE INF5130 : Algorithmique}

\bigskip
\bigskip
\bigskip

\textbf{\LARGE Devoir 1}

\vspace{0.5em}

\bigskip
\bigskip

\large

Hiver 2023

\bigskip

\normalsize

\bigskip
\bigskip

\end{center}





\rm

\bigskip
\bigskip

Nom : \underline{OUEYEYA} \hfill{}  Code permanent : \underline{OUEG82330306} 

\bigskip
\bigskip

Nom : \underline{Gaëtan} \hfill{}Code permanent : \underline{OUEG82330306} 


\newpage


\underline{Exercice 1} 
\begin{enumerate}[label=\Roman*.]
\item 

  $Tab=$ \begin{tabular}{|c|c|c|c|c|c|c|c|}
\hline
     1  & 2 & 0  &0 & 0 &4 & 2 &0 \\
    \hline
\end{tabular}






\item 
 

\item 



\item 

\begin{enumerate}[label=\alph*)]

     \item 
     
     \item 

\end{enumerate}

\end{enumerate}

\bigskip

\underline{Exercice 2} 

\begin{enumerate}[label=\alph*)]

     \item $$ \sum_{i=2}^{n-1} \sum_{j=3}^{i+3}\left( 36j^2+24j+18i \right)  \textrm{ pour tout } n \in \mathbb{N}, n\geq 3.$$
     
     
     \item $$ \sum_{i=5}^{n} \sum_{j=0}^{i}\sum_{k=0}^{j}3^k  \textrm{ pour tout } n \in \mathbb{N}, n\geq 5.$$
  

\end{enumerate}

\bigskip

\underline{Exercice 3} On suppose que $f(n) = \omega\left(g(n)\right)$. Démontrer \textbf{à l'aide des définitions formelles} que $5g(n) +3 f(n) = \omega \left(2g(n)\right)$.




\bigskip

\underline{Exercice 4}
On suppose que $f(n) = O\left(h(n)\right)$ et $g(n) = \Omega \left(1\right)$. Démontrer \textbf{à l'aide des définitions formelles} que $7f(n) =O \left( h(n) g(n)\right)$.

\bigskip


\underline{Exercice 5} 

On considère l’équation de récurrence $T(n)=4T(n-1) + 5T(n-2) +1$ pour tout entier $n \geq 3$, $T(1)=0$, $T(2)=1$.

\begin{enumerate}[label=\alph*)]

     \item Calculer $T(3)$, $T(4)$ et $T(5)$ à l'aide de l'équation de récurrence.
     
     
     \item Démontrer \textbf{par récurrence} que $T(n)\geq  5^{n-2}$ pour tout entier $n \geq 2$.

\end{enumerate}

\bigskip

\underline{Exercice 6} 



\begin{enumerate}[label=\alph*)]
   \item $T(n)=8 T \left( \lceil{\frac{n}{4}} \rceil\right) + 5 n \, \sqrt{n} $
     
     \item $T(n)=9T \left( \lceil{\frac{n}{3}} \rceil\right) + 7n^6 $

\item $T(n)=6 T \left( \lceil{\frac{n}{36}} \rceil\right) + \sqrt[3]{n} \lg(n) $

\end{enumerate}

\bigskip


\underline{Exercice 7} 
$$   n \lg^2(5n), \sqrt[3]{n^7}, \frac{3n^5+5n}{2n^2+1},2^{2 \log_{16}(n)} , \sqrt{n^2 \lg^5\left(\sqrt{n}\right)}, 5^{\lg(n)},n \lg(n^3),\frac{\sqrt{n}}{\lg(n)}$$

\bigskip


\underline{Exercice 8} 

$T(n)=\ n^2\, 32^{ 2\log_4(n)} \lg^3\left(\sqrt[3]{n}\right) \, \log_5\left(\frac{n^4}{\sqrt{n}} \right) $. 

\end{document}





