  \documentclass[11pt]{article}
 \setlength\headheight{15pt} 
 \usepackage[left=2cm,top=3cm,right=2cm,bottom=3cm]{geometry}
  \usepackage[utf8]{inputenc}
    \usepackage{setspace}
    \usepackage{tabto}
  \usepackage{enumitem}
  \usepackage{fancyhdr}
  \pagestyle{fancy}
\usepackage{amsfonts,amsmath,amssymb}
\usepackage{graphicx}


\begin{document}

\thispagestyle{empty}

\begin{center}
\textbf{\Large Université du Qu\'ebec \`a Montr\'eal}

\bigskip
\bigskip

\textbf{\LARGE INF5130 : Algorithmique}

\bigskip
\bigskip
\bigskip

\textbf{\LARGE Devoir 1}

\vspace{0.5em}

\bigskip
\bigskip

\large

Hiver 2023

\bigskip

\normalsize

\bigskip
\bigskip

\end{center}





\rm

\bigskip
\bigskip

Nom : \underline{OUEYEYA} \hfill{}  Code permanent : \underline{OUEG82330306} 

\bigskip
\bigskip

Nom : \underline{Gaëtan} \hfill{}Code permanent : \underline{OUEG82330306} 


\newpage


\underline{Exercice 1} 
\begin{enumerate}[label=\Roman*.]
\item 

  $Tab=$ \begin{tabular}{|c|c|c|c|c|c|c|c|}
\hline
     1  & 2 & 0  &0 & 0 &4 & 2 &0 \\
    \hline
\end{tabular}

\bigskip

$NZ(Tab) = 1 + 1 + 0 + 0 + 0 + 1 + 1 + 0 = 4$


\item 
 Le pire cas de cet algorithme correspond à un tableau $Tab$ contenant uniquement des éléments non nuls, car les conditions des si sont toujours vérifiées. Ainsi, la valeur de $NZ(Tab)$ dans le pire cas est $\frac{n(n+1)}{2}$, où n est la taille du tableau.

\item 
Le meilleur cas de cet algorithme correspond à un tableau $Tab$ ne contenant que des zéros, car les conditions du si ne sont jamais vérifiées. Ainsi, la valeur de $NZ(Tab)$ dans le meilleur cas est 0.


\item 

\begin{enumerate}[label=\alph*)]

     \item 
     
     \begin{equation*}
          T(n) = c_1 + c_9 + c_2(n+1)+ \sum_{i=1}^{n}\left(c_3(i+1)+(c_4+c_8)i\sum_{j=1}^{i}(i-j+1)c_5+(i-j)(c_6+c_7)\right)
     \end{equation*}
        
         \begin{multline*}
         = c_1 + c_9 + c_2(n+1)+ c_3\cdot \left(\frac{n(n+1)}{2}+n\right)+(c_4+c_8)\left(\frac{n(n+1)}{2}+n\right)+\\
         \sum_{i=1}^{n}\left(i(i+1)c_5-\frac{i(i+1)}{2}c_5-i^{2}(c_6+c_7)-\frac{i(i+1)}{2}(c_6+c_7) \right)
         \end{multline*}

         \begin{multline*}
           = c_1 + c_9 + c_2(n+1)+ c_3n+ (c_3+c_4+c_8)\left(\frac{n(n+1)}{2}\right)+\\
           \sum_{i=1}^{n}i^{2}(c_5+c_6+c_7)+c_5i-\frac{i(i+1)}{2}(c_5+c_6+c_7) 
         \end{multline*}
         
         \begin{multline*}
            = c_1 + c_9 + c_2(n+1)+ c_3n+ (c_3+c_4+c_8)\left(\frac{n(n+1)}{2}\right)+ (c_5+c_6+c_7)\left(\frac{n(n+1)(2n+1)}{6}\right)+\\
            c_5\left(\frac{n(n+1)}{2}\right)-\frac{c_5+c_6+c_7}{2}\left(\frac{n(n+1)(2n+1)}{6}\right)-\left(\frac{n(n+1)}{2}\right)\frac{c_5+c_6+c_7}{2}
         \end{multline*}

        \begin{multline*}
            = c_1 + c_9 + c_2(n+1)+ c_3n+ (c_3+c_4+c_8)\left(\frac{n(n+1)}{2}\right)+ \\
            \frac{(c_5+c_6+c_7)}{2}\left(\frac{n(n+1)(2n+1)}{6}\right)+            \frac{(c_5-c_6-c_7)}{4}\left(\frac{n(n+1)}{2}\right)
         \end{multline*}

        \begin{multline*}
            = c_1 + c_9 + c_2(n+1)+ c_3n+ \left(c_3+c_4+c_8+\frac{(c_5-c_6-c_7)}{4}\right)\left(\frac{n(n+1)}{2}\right)+ \\
            \frac{(c_5+c_6+c_7)}{2}\left(\frac{(n^{2}+n)(2n+1)}{6}\right)
         \end{multline*}
        \begin{multline*}
            = c_1 + c_9 + c_2(n+1)+ c_3n+ \left(\frac{c_3+c_4+c_8}{2}+\frac{(c_5-c_6-c_7)}{8}\right)(n^{2}n)+ \\
            \frac{(c_5+c_6+c_7)}{12}(2n^{3}+3n^{2}+n)
         \end{multline*}
        \begin{multline*}
            = c_1 + c_9 + c_2(n+1)+ c_3n+ \left(\frac{4c_3+4c_4+4c_8+c_5-c_6-c_7}{8}\right)(n^{2}n)+ \\
            \frac{(c_5+c_6+c_7)}{12}(2n^{3}+3n^{2}+n)
         \end{multline*}
        \begin{multline*}
            = \frac{(c_5+c_6+c_7)}{6}n3+\frac{(c_5+c_6+c_7)}{12}n^{2}+\left(\frac{4c_3+4c_4+4c_8+c_5-c_6-c_7}{8}\right)n^{2}\\
            +\left(\frac{4c_3+4c_4+4c_8+c_5-c_6-c_7}{8}\right)n+\frac{(c_5+c_6+c_7)}{12}n+c_3n+c_2n+c_2+c_1 + c_9  
         \end{multline*}
        \begin{multline*}
            = \frac{(c_5+c_6+c_7)}{6}n3+\frac{(12c_3+12c_4+12c_8+5c_5-c_6-c_7)}{24}n^{2}+\\
            +\left(\frac{4c_3+4c_4+4c_8+c_5-c_6-c_7}{8}\right)n+\frac{(12c_3+12c_2+c_5+c_6+c_7)}{12}n++c_2+c_1 + c_9  
         \end{multline*}
        \begin{multline*}
            = \frac{(c_5+c_6+c_7)}{6}n3+\frac{(12c_3+12c_4+12c_8+5c_5-c_6-c_7)}{24}n^{2}+\\
            +\left(\frac{36c_3+24c_2+12c_4+12c_8+5c_5-c_6-c_7}{24}\right)n+c_2+c_1 + c_9  
         \end{multline*}
        
     \item 
     On doit supprimer $c_7$ donc :
             \begin{multline*}
            = \frac{(c_5+c_6)}{6}n3+\frac{(12c_3+12c_4+12c_8+5c_5-c_6)}{24}n^{2}+\\
            +\left(\frac{36c_3+24c_2+12c_4+12c_8+5c_5-c_6}{24}\right)n+c_2+c_1 + c_9  
         \end{multline*}

    \end{enumerate}

\end{enumerate}

\bigskip

\underline{Exercice 2} 

\begin{enumerate}[label=\alph*)]

     \item $$ \sum_{i=2}^{n-1} \sum_{j=3}^{i+3}\left( 36j^2+24j+18i \right)  \textrm{ pour tout } n \in \mathbb{N}, n\geq 3.$$
     
     \begin{align*}
        &=\sum_{i=2}^{n-1} \left[ \sum_{j=3}^{i+3}\left( 36j^2+24j+18i \right) \right]\\
        &=\sum_{i=2}^{n-1} \left[ \sum_{j=3}^{i+3}\left( 36j^2 \right) +\sum_{j=3}^{i+3}\left( 24j \right) +\sum_{j=3}^{i+3}\left( 18i \right) \right]\\
        &=\sum_{i=2}^{n-1} \left[ 36\sum_{j=3}^{i+3}j^2 + 24\sum_{j=3}^{i+3}j + 18i\sum_{j=3}^{i+3} 1 \right]\\
        &=\sum_{i=2}^{n-1} \left[ 36\left(\sum_{j=1}^{i+3}j^2 -(2^{2}+2^{1})\right)+ 24\left(\sum_{j=1}^{i+3}j - (2+1)\right) + 18i(i+3-3+1) \right]\\
        &=\sum_{i=2}^{n-1} \left[ 36\left(\frac{(i+3)(i+4)(2(i+3)+1)}{6} -5\right)+ 24\left(\frac{(i+3)(i+4)}{2} - 3\right) + 18i^{2} +18i \right]\\
        &=\sum_{i=2}^{n-1} \left[ 36\left(\frac{(i^{2}+7i+12)(2i+7)}{6} -5\right)+ 24\left(\frac{(i^{2}+7i+12)}{2} - 3\right) + 18i^{2} +18i \right]\\
        &=\sum_{i=2}^{n-1} \left[ 36\left(\frac{2i^{3}+21i^{2}+73i}{6} + 9\right)+ 24\left(\frac{i^{2}+7i}{2} + 3\right) + 18i^{2} +18i \right]\\
        &=\sum_{i=2}^{n-1} \left[ 6\left(2i^{3}+21i^{2}+73i \right)+ 9\times 36 + 12\left(i^{2}+7i \right)+ 3\times 24 + 18i^{2} +18i \right]\\
        &=\sum_{i=2}^{n-1} \left[ 12i^{3}+126i^{2}+438i+ 324 + 12i^{2}+84i+72+18i^{2} +18i \right]\\
        &=\sum_{i=2}^{n-1}12i^{3}+156i^{2}+540i+ 396\\   
        &=\sum_{i=2}^{n-1}12i^{3}+\sum_{i=2}^{n-1}156i^{2}+\sum_{i=2}^{n-1}540i+ \sum_{i=2}^{n-1}396\\  
        &=12\sum_{i=2}^{n-1}i^{3}+156\sum_{i=2}^{n-1}i^{2}+540\sum_{i=2}^{n-1}i+ 396(n-1-2+1)\\ 
        &=12\sum_{i=2}^{n-1}i^{3}+156\sum_{i=2}^{n-1}i^{2}+540\sum_{i=2}^{n-1}i+ 396(n-1-2+1)\\ 
        &=12\sum_{i=2}^{n-1}(i^{3}-1)+156\sum_{i=2}^{n-1}(i^{2}-1)+540\sum_{i=2}^{n-1}(i-1)+ 396n-792\\
        &=12\left(\frac{(n-1)^{2}(n)}{4}\right)-12+156\left(\frac{(n-1)(n)(2(n-1)+1)}{6}\right)-156+540\left(\frac{(n-1)(n)}{2}\right)-540+396n-792\\
        &=3(n^{2}-2n+1)n^{2})-12+26((n^{2}-n)(2n-1))-156+270n^{2}-270n-540+396-792\\
        &=3n^{4}-6n^{3}+3n^{2}-12+52n^{3}-26n^{2}-52n^{2}+270n^{2}+26n-156+270n-270n-540+396n-792\\
        &=3n^{4}+46n^{3}+195n^{2}-152n-1500
     \end{align*}
     
     \item $$ \sum_{i=5}^{n} \sum_{j=0}^{i}\sum_{k=0}^{j}3^k  \textrm{ pour tout } n \in \mathbb{N}, n\geq 5.$$

    \begin{align*}
        &=\sum_{i=5}^{n} \sum_{j=0}^{i}\frac{3^{j+1}-1}{3-1}\\
        &=\sum_{i=5}^{n} \sum_{j=0}^{i}\frac{1}{2}\left(3^{j}+1\right)\\
        &=\sum_{i=5}^{n} \frac{1}{2}\sum_{j=0}^{i}3^{j}+1\\
        &=\sum_{i=5}^{n} \frac{1}{2}\left(\frac{3^{i+2}-1}{2}\right) -\frac{1}{2}\left(i+1\right)\\
        &=\sum_{i=5}^{n} \frac{1}{4}\left(3^{i+2}-2i-5\right)\\
        &=\frac{1}{4}\left(\frac{3^{n+3}-1}{2}\right)-\frac{1}{4}(3^{2}+3^{3}+3^{4}+3^{5}+3^{6})-\frac{1}{2}\left(\frac{n(n+1)}{2}\right)+\frac{10}{2}-\frac{5}{4}(n-4)\\
        &=\frac{1}{3}\left(3^{n+3}-1\right)-\frac{1}{4}(1089)-\frac{1}{4}(n^{2}+n)+5-\frac{5n}{4}+5\\
        &=\frac{1}{8}\left(3^{n+3}-1-2178-2n^{2}-2n+40-10n+40\right)\\
        &=\frac{1}{8}\left(3^{n+3}-2n^{2}-12n-2099\right)\\
    \end{align*}
  
\end{enumerate}

\bigskip

\underline{Exercice 3} 

    On suppose que $f(n) = \omega\left(g(n)\right)$.

    Pour toute constante strictement positive $c_0$, il existe $n_0>0$ telle que pour tout $n > n_0$,
    
    $0\leq c_0\cdot g(n)\leq f(n)$.
    Ainsi,

    $0\leq 6\cdot c_0\cdot g(n)\leq 6\cdot f(n)$ 
    
    $0\leq \frac{6\cdot c_0\cdot g(n)}{2}\leq \frac{6}{2}\cdot f(n)$

    $0 \leq \frac{2\cdot 3\cdot c_0}{2}\cdot g(n)\leq 3\cdot f(n)$

    $0 \leq \left(\frac{ 3\cdot c_0}{2}+5\right)\cdot 2g(n)\leq 3\cdot f(n) + 5\cdot f(n)$
    
   \bigskip 

    En posant c = $\frac{2\cdot 3\cdot c0}{2}+5$,

    \begin{center}
        pour tout $n > n_0$, $0 \leq c\cdot 2g(n)\leq 3\cdot f(n) + 5\cdot f(n)$,
    \end{center}
    ce qui démontre que $f(n) = \omega\left(g(n)\right)$.


\bigskip

\underline{Exercice 4}
On suppose que $f(n) = O\left(h(n)\right)$ et $g(n) = \Omega \left(1\right)$. Démontrer \textbf{à l'aide des définitions formelles} que $7f(n) =O \left( h(n) g(n)\right)$.

\bigskip

    On suppose que $f(n) = O\left(h(n)\right)$ et $g(n) = \Omega \left(1\right)$.
    
    Si $f(n) = O\left(h(n)\right)$ alors il existe deux constantes strictement positives $c_1$ et $n_1$ telles que
    \begin{center}
        $0\leq f(n) \leq c_1 \cdot h(n)$, pour tout $n \geq n_1$.
        
        $0 \leq 7\cdot f(n) \leq 7\cdot c_1\cdot h(n)$
    \end{center}
    
    

    Donc avec $7\cdot c1 = d$, $0 \leq 7\cdot f(n) \leq d\cdot h(n)$.

    \bigskip

    De plus, si $g(n) = \Omega \left(1\right)$, alors il existe deux constantes strictement positives $c_2$ et $n_2$ tel que

    $0\leq 1\cdot c_2 \leq g(n)$ pour tout $n \geq n_2$.

    $0 \leq c_2\cdot h(n) \leq g(n)\cdot h(n)$

    \bigskip

    Ainsi, pour tout $n \geq max(n_1,n_2)$ et $c = min(d, c_2)$, 
    
    $0 \leq 7\cdot f(n) \leq \left( h(n) g(n)\right)$ et $7f(n) = O \left( h(n) g(n)\right)$.

    Donc $7f(n) =O \left( h(n) g(n)\right)$.
    
\bigskip


\underline{Exercice 5} 

On considère l’équation de récurrence $T(n)=4T(n-1) + 5T(n-2) +1$ pour tout entier $n \geq 3$, $T(1)=0$, $T(2)=1$.

\begin{enumerate}[label=\alph*)]

     \item Calculer $T(3)$, $T(4)$ et $T(5)$ à l'aide de l'équation de récurrence.
         \begin{align*} 
            T(3) &= 4T(2) + 5T(1)\\
            T(3) &= 8 
         \end{align*}
        \begin{align*} 
            T(4) &= 4T(3) + 5T(2)\\
            T(4) &= 4\cdot 8 + 5\cdot 1\\
            T(4) &= 37
        \end{align*}
        \begin{align*} 
            T(5) &= 4T(4) + 5T(3)\\
            T(5) &= 4\cdot 37 + 5\cdot 8\\
            T(5) &= 188
        \end{align*}
     
     \item Démontrer \textbf{par récurrence} que $T(n)\geq  5^{n-2}$ pour tout entier $n \geq 2$.

        \textbf{Initialisation :}
        
        Si $n=2$, $T(2)=1$ et $5^{2-2} \Leftrightarrow 5^0 \leq 1$.

        Si $n=3$, $T(3)=8$ et $5^{3-2} \Leftrightarrow 5 \leq 8$.

        L'initialisation est donc vérifiée.

        \bigskip

        \textbf{Récurrence :}

        On suppose la proposition vraie au rang $k = n$ : $T(n)\geq  5^{n-2}$ pour tout entier $n \geq 2$. On veut montrer qu'elle est vraie au rang $k = n+1$ :$T(n+1)\geq  5^{n-1}$.

        On pose que $T(n)\geq  5^{n-2}$ et $T(n-1)\geq  5^{n-3}$.

        \begin{align*}
            T(n+1) &= 4\cdot T(n) + 5\cdot T(n-1)\\
            &\geq 4\cdot 5^{n-2} + 5\cdot 5^{n-3}\\
            &\geq 5^{n-1}\cdot(4 \times 5^{-1} + 5\times 5^{-2})\\
            &\geq 5^{n-1}\cdot(4 \times 0.2 + 5\times 0.04)\\
            &\geq 5^{n-1}\cdot(0.8 + 0.2)\\
            &\geq 5^{n-1}
        \end{align*} 

        $T(n+1)\geq  5^{n-1}$ est donc vraie et la proposition est donc récurrente.
        
        \bigskip

        \textbf{Conclusion :}

        La proposition est initialisée et récurrente donc pour tout $n \geq 2$, $T(n) = 5^{n-2}$.   
\end{enumerate}

\bigskip

\underline{Exercice 6} 



\begin{enumerate}[label=\alph*)]
   \item $T(n)=8 T \left( \lceil{\frac{n}{4}} \rceil\right) + 5 n \, \sqrt{n} $
   
   $a=8$, $b=4$ et $f(n) = 5n\sqrt{n} = 5n^{\frac{3}{2}} = \Theta(n^{\frac{3}{2}})$

   $\log_b(a) = \log_4(8) = \frac{3}{2}$.

    Le cas 2 s'applique donc $T(n)=\left(n^{\log_4(8)}\cdot \lg(n)\right)$.
    
     \item $T(n)=9T \left( \lceil{\frac{n}{3}} \rceil\right) + 7n^6 $

    $a=9$, $b=3$ et $f(n) = 7n^{6}$

    $\log_b(a) = \log_3(9) = 2$ et $n^{\log_b(a)}=n^{2}$.

    Le cas 3 s'applique donc si on prend $\epsilon=4$, $f(n) = \Omega(n^{\log_3(9)+4}$ donc $T(n)=\Theta(f(n))$.
    
    De plus, on vérifie l'hypothèse que $a\cdot f\left(\frac{n}{b}\right)$,
    
    \begin{align*}
        9\cdot f\left(\frac{n}{3}\right) &= 9 \times 7\left(\frac{n}{3}\right)^{6}\\
        &= \frac{63n^{6}}{729}\\
        &= \frac{7n^{6}}{81}
    \end{align*}
    
    On peut donc prendre $c=\frac{1}{81}<1$.
    
    \item $T(n)=6 T \left( \lceil{\frac{n}{36}} \rceil\right) + \sqrt[3]{n} \lg(n) $
    $a = 6$, $b=36$ et $f(n)=\sqrt[3]{n} \lg(n)$.

    $\log_b(a) = \log_{36} (6) = \frac{1}{2} et n^{\log_b(a)}=n^{\frac{1}{2}}$.
    
    Le cas 1 s'applique donc si on prend $\epsilon=\frac{1}{6}$, $f(n) = \Omega\left(n^{\log_{36}(6)-\frac{1}{6}}\right)$ donc $T(n)=\Theta\left(n^{\log_{36}(6)}\right)$.

\end{enumerate}

\bigskip


\underline{Exercice 7} 

$$\frac{\sqrt{n}}{\lg(n)}, 2^{2 \log_{16}(n)}, \sqrt{n^2 \lg^5 \left(\sqrt{n}\right)}, n \lg^2(5n),n \lg(n^3), \sqrt[3]{n^7}, 5^{\lg(n)}, \frac{3n^5+5n}{2n^2+1}$$

\bigskip


\underline{Exercice 8} 
\begin{align*}
   T(n)&=\ n^2\, 32^{ 2\log_4(n)} \lg^3\left(\sqrt[3]{n}\right) \, \log_5\left(\frac{n^4}{\sqrt{n}} \right) \\
   &= n^{2}\cdot \left(32^{\log_4(n)}\right)^{2}\cdot \left(\lg(\sqrt[3]{n})\right)^{3}\cdot \left(\log_5(n^{4})-\log_5(\sqrt{n})\right)\\
   &= n^{2} \cdot \left(n^{\log_4(32)}\right)^{2}\cdot \left(\frac{1}{3}\lg(n)\right)^{3}\cdot \left(4\log_5(n) - \frac{1}{2}\log_5(n)\right)\\
   &= n^{2} \cdot \left(n^{5/2}\right)^{2} \cdot \left(\frac{1}{3}\lg(n)\right)^{3} \cdot \left(\frac{7}{2}\log_5(n)\right)\\
   &=n^{2} \cdot n^{5} \cdot \frac{7}{27\times2} \cdot\lg^{3}(n) \cdot \log_5(n)\\
   &=\frac{7}{54}\cdot n^{7}\cdot\lg^{3}(n) \cdot \frac{\lg(n)}{\log_5}\\
   &=\frac{7}{54\times\log_5}\cdot n^{7}\cdot\lg^{4}(n)
\end{align*}
 
Ainsi, $T(n) =\frac{7}{54\times\log_5}\cdot n^{7}\cdot\lg^{4}(n)$.


\end{document}





