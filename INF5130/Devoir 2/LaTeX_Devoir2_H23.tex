 \documentclass[11pt]{article}
 % \usepackage[utf8]{inputenc}
  \usepackage{enumitem}
\usepackage{amsfonts,amsmath,amssymb}
\usepackage[left=2cm,top=3cm,right=2cm,bottom=3cm]{geometry}
\usepackage{graphicx}
%\usepackage{tikz}
%\usetikzlibrary{graphdrawing.trees}
%\usetikzlibrary{arrows, calc, positioning}
%\usepackage{float, subcaption}

\usepackage{forest}


\begin{document}

\thispagestyle{empty}

\begin{center}
\textbf{\Large Université du Qu\'ebec \`a Montr\'eal}

\bigskip
\bigskip

\textbf{\LARGE INF5130 : Algorithmique}

\bigskip
\bigskip
\bigskip

\textbf{\LARGE Devoir 2}

\vspace{0.5em}

\bigskip
\bigskip

\large

Hiver 2023

\bigskip

\normalsize

\bigskip
\bigskip

\end{center}





\rm

\bigskip
\bigskip

Nom : \underline{????} \hfill{}  Code permanent : \underline{????} 

\bigskip
\bigskip

Nom : \underline{????} \hfill{}Code permanent : \underline{????} 


\newpage

\thispagestyle{empty}

\underline{Exercice 1} \textbf{(25 points)}

 \begin{center}
\begin{tabular}{ |l| c c c c c c c|}
 \hline 
 $k$ & $1$ & $2$ & $3$ & $4$ & $5$ & $6$ & $7$\\ 
 \hline 
 $p_k$ & $0,\!21$ & $0,\!01$ & $0,\!05$ & $0,\!12$ & $0,\!18$ & $0,\!03$ & $0,\!4$ \\
  \hline 
\end{tabular}
\end{center}

\begin{center}


\forestset{
    treenode/.style = {rectangle, draw, base=top},
    payoff/.style   = {align=center, base=top}
    }

\begin{forest}
  for tree={child anchor=north}
   [ 	[\texttt{$5$:$0,\!18$},no edge,treenode 
      		[\texttt{$2$:$0,\!25$},treenode,edge 
        		[\texttt{$1$:$0,\!1$},treenode,edge ]
      			[\texttt{$4$:$0,\!12$},treenode,edge 
      				[\texttt{$3$:$0,\!05$},treenode,edge ]
      				[\texttt{}, no edge]
      			]
      		]
      		[\texttt{$6$:$0,\!3$},treenode,edge ]
    	]
    ]
  \end{forest}
\end{center}



\underline{Exercice 2} \textbf{(25 points)}


\begin{center}
\begin{tabular}{ |c| c c c c c c c c  c c |}
 \hline 
Tâche & 1 & 2 & 3 & 4  & 5 & 6 & 7 & 8 & 9 & 10 \\ 
 \hline 
Échéance & 3 & 2 & 2 & 3  & 4 & 5 & 1 & 5 & 6 & 8  \\
 Pénalité & 95 & 85 & 55 & 60  & 50 & 45 & 40 & 30 & 20 & 10 \\ 

  \hline 
\end{tabular}
\end{center}


\bigskip
\bigskip


\underline{Exercice 3} \textbf{(25 points)}
 
$$\left(\left( x_1 \rightarrow   x_2 \right) \land   x_1 \right)   \lor \lnot x_2 .$$



\forestset{
    treenode/.style = {circle, draw, base=top},
    payoff/.style   = {align=center, base=top}
    }

 \begin{forest}
  for tree={child anchor=north,minimum size=8 mm}
   [ 	[$\lnot$, edge label={node[midway, right]{$y_1$}},treenode  
      		[$\lnot x_2$]
      		[$\lor$,treenode,edge label={node[below = -0.15cm,right]{$y_2$}}
        		[$\to$,treenode,edge label={node[below = -0.15cm,left]{$y_3$}}
					[$\lnot x_2$]
        			[$ x_1$]        		
        		]
        		[$\land$,treenode,edge label={node[below = -0.15cm,right]{$y_4$}}
        			[$ x_3$]
        			[$\lnot x_4$]  
        		]
      		]
    	]
    ]
  \end{forest}

\bigskip
\bigskip

\newpage

\underline{Exercice 4} \textbf{(25 points)}





Par exemple pour $m=3$ et $n=3$, si nous avons en entrée la matrice de salaire $S=\left( \begin{matrix}
0 & 0 & 0   \\
15 & 20 & 23  \\
40 & 35 & 45  \\
62 & 50 & 55  \\
\end{matrix}  \right)$, cela signifie que l'entreprise $1$ paye $62$ \$ pour $3$ heures et que l'entreprise 3 paye $45$ \$ pour $2$ heures. On peut dans ce cas calculer assez facilement les matrices $SM$ et $NH$ en considérant toutes les combinaisons possibles et on obtient : 




$$SM=\left( \begin{matrix}
0 & 0 & 0  \\
15 & 20 & 23  \\
40 & 40 & 45  \\
 62& 62 & 65  \\
\end{matrix}  \right) \,\,\,  \,\,\,  \,\,\,  \,\,\, NH=\left( \begin{matrix}
0 & 0 & 0   \\
 1& 1 & 1  \\
2 &0 & 2  \\
3 & 0 & 2  \\
\end{matrix}  \right).  $$

Le revenu maximal est ainsi de $65$ \$ pour 2 heures de travail dans l'entreprise $3$ et $1$ heure de travail dans l'entreprise $2$. De manière plus générale, les matrices $SM$ et $NH$ peuvent être calculées selon le schéma dynamique suivant.

\smallskip

\underline{\bf Initialisation} : \hspace{0.44 cm} Pour tout entier  $1\leq j \leq n$, $SM[0][j]=NH[0][j]=0$.

\smallskip

\hspace{3.32 cm} Pour tout entier  $1\leq i \leq m$, $SM[i][1]=S[i][1]$ et $NH[i][1]=i$.


\medskip

\underline{\bf Schéma récursif} : Pour tout entier  $1 \leq i \leq m$ et pour tout entier  $2 \leq j \leq n$, on a :

\smallskip

\hspace{3.32 cm} $SM[i][j]=\max_{0\leq k \leq i}\left(SM[i-k][j-1]+S[k][j]  \right)$,

\smallskip


\hspace{3.32 cm} $NH[i][j]$ est égal à l'indice $k$ qui donne la valeur maximale dans la formule

 \hspace{3.32 cm} ci-dessus.

\bigskip
Le coefficient $SM[m][n]$ est alors égal au revenu maximal, et le coefficient $NH[m][n]$ est égal au nombre d'heures qu'il faut travailler dans l'entreprise $n$ pour obtenir ce revenu maximal.

\bigskip

\textbf{\underline{Questions}} : 

\begin{enumerate}[label=\Roman*.]

\item Déterminer les matrices $SM$ et $NH$ \textbf{en utilisant les formules ci-dessus},  si on a en entrée la matrice  suivante (pour $m=5$ heures de travail et $n=4$ entreprises):



$$S=\left( \begin{matrix}
0& 0& 0 & 0   \\
15& 20& 18 & 25   \\
30& 40& 38 & 35   \\
45& 54& 59 & 47   \\
60& 65& 68 & 69   \\
75& 70& 78 & 80   \\
\end{matrix}  \right).$$

\textbf{Vous devez expliciter tous les détails de vos calculs.}

\item En déduire le revenu maximal et le nombre d'heures qu'il faut travailler dans chacune des entreprises pour obtenir ce revenu maximal.

\textbf{Vous devez notamment expliquer comment vous obtenez le nombre d'heures travaillées dans chaque entreprise à partir de la matrice $NH$.}

\end{enumerate}




\end{document}





