 \documentclass[11pt]{article}
 % \usepackage[utf8]{inputenc}
  \usepackage{enumitem}
\usepackage{amsfonts,amsmath,amssymb}
\usepackage[left=2cm,top=3cm,right=2cm,bottom=3cm]{geometry}
\usepackage{graphicx}
%\usepackage{tikz}
%\usetikzlibrary{graphdrawing.trees}
%\usetikzlibrary{arrows, calc, positioning}
%\usepackage{float, subcaption}

\usepackage{forest}


\begin{document}

\thispagestyle{empty}

\begin{center}
\textbf{\Large Université du Qu\'ebec \`a Montr\'eal}

\bigskip
\bigskip

\textbf{\LARGE INF5130 : Algorithmique}

\bigskip
\bigskip
\bigskip

\textbf{\LARGE Devoir 2}

\vspace{0.5em}

\bigskip
\bigskip

\large

Hiver 2023

\bigskip

\normalsize

\bigskip
\bigskip

\end{center}





\rm

\bigskip
\bigskip

Nom : \underline{Oueyeya Gaëtan} \hfill{}  Code permanent : \underline{OUEG82330306} 


\newpage

\thispagestyle{empty}

\underline{Exercice 1} \textbf{(25 points)}

 \begin{center}
\begin{tabular}{ |l| c c c c c c c|}
 \hline 
 $k$ & $1$ & $2$ & $3$ & $4$ & $5$ & $6$ & $7$\\ 
 \hline 
 $p_k$ & $0,\!21$ & $0,\!01$ & $0,\!05$ & $0,\!12$ & $0,\!18$ & $0,\!03$ & $0,\!4$ \\
  \hline 
\end{tabular}
\end{center}

\bigskip

\begin{center}
    $C[1; 1] = p1 = 0,21$ \quad $C[2; 2] = p2 = 0,01$ \quad $C[3; 3] = p3 = 0,05$\\ \smallskip
    $C[4; 4] = p4 = 0,12$ \quad $C[5; 5] = p5 = 0,18$ \quad $C[6; 6] = p6 = 0,03$\\ \smallskip
    $C[7; 7] = p7 = 0,4$
\end{center}

\bigskip

$C[1; 2] = min(\textbf{0 + 0.1} ; 0.21 + 0) + 0.21 + 0.01 = 0.23$\\ \smallskip
$C[2; 3] = min(0 + 0.5 ; \textbf{0.01 + 0}) + 0.01 + 0.05= 0.07$\\ \smallskip
$C[3; 4] = min(0 + 0.12 ; \textbf{0.05 + 0}) + 0.05 + 0.12= 0.22$\\ \smallskip
$C[4; 5] = min(0 + 0.18 ; \textbf{0.12 + 0}) + 0.12 + 0.18= 0.42$\\ \smallskip
$C[5; 6] = min(\textbf{0 + 0.03} ; 0.18 + 0) + 0.18 + 0.03= 0.24$\\ \smallskip
$C[6; 7] = min(0 + 0.4 ; \textbf{0.03 + 0}) + 0.4 + 0.03= 0.46$\\ \smallskip

\bigskip

$C[1; 3] = min(\textbf{0 + 0.07} ; 0.21 + 0.05; 0.23 + 0) + 0.21 + 0.01 + 0.05= 0.34$\\ \smallskip
$C[2; 4] = min(0 + 0.22 ; 0.1 + 0.12; \textbf{0.07 + 0}) + 0.01 + 0.05 + 0.12= 0.25$\\ \smallskip
$C[3; 5] = min(0 + 0.42 ; 0.05 + 0.18; \textbf{0.22 + 0}) + 0.05 + 0.12 + 0.18= 0.57$\\ \smallskip
$C[4; 6] = min(0 + 0.24 ; \textbf{0.12 + 0.03}; 0.42 + 0) + 0.12 + 0.18 + 0.03= 0.48$\\ \smallskip
$C[5; 7] = min(0 + 0.46 ; 0.18 + 0.4 ;\textbf{0.24 + 0}) + 0.18 + 0.03 + 0.4= 0.85$\\

\bigskip

$C[1; 4] = min(\textbf{0 + 0.25} ; 0.21 + 0.22; 0.23 + 0.12;  0.34 + 0) + 0.21 + 0.01 + 0.05 + 0.12= 0.64$\\ \smallskip
$C[2; 5] = min(0 + 0.57 ; 0.01 + 0.42; \textbf{0.07 + 0.18}; \textbf{0.25 + 0}) + 0.01 + 0.05 + 0.12 + 0.18= 0.61$\\ \smallskip
$C[3; 6] = min(0 + 0.42 ; 0.05 + 0.24; \textbf{0.22 + 0.03}; 0.57 + 0) + 0.05 + 0.12 + 0.18 + 0.03= 0.63$\\ \smallskip
$C[4; 7] = min(0 + 0.85 ; 0.12 + 0.46; 0.42 + 0.4 ;\textbf{ 0.48 + 0}) + 0.12 + 0.18 + 0.03 + 0.4 = 1.21$\\

\bigskip

$C[1; 5] = min(0 + 0.61 ; 0.21 + 0.57; 0.23 + 0.42; \textbf{0.34 + 0.18} ; 0.64+0) + 0.21 + 0.01 + 0.05 + 0.12 + 0.18 = 1.09$\\ \smallskip
$C[2; 6] = min(0 + 0.63 ; 0.01 + 0.48; 0.07 + 0.24; \textbf{0.25 + 0.03}; 0.61 +0) + 0.01 + 0.05 + 0.12 + 0.18 +0.03= 0.67$\\ \smallskip
$C[3; 7] = min(0 + 1.21 ; 0.05 + 0.85; 0.22 + 0.45; 0.57+0.4; \textbf{0.63 + 0}) + 0.05 + 0.12 + 0.18 + 0.03+ 0.4= 1.41$\\

\bigskip

$C[1; 6] = min(0 + 0.67 ; 0.21 + 0.63; 0.23 + 0.48; \textbf{0.34 + 0.24} ; 0.64+0.03; 1.09 + 0) + 0.21 + 0.01 + 0.05 + 0.12 + 0.18 + 0.03= 1.18$\\ \smallskip
$C[2; 7] = min(0 + 1.41 ; 0.01 + 1.21; 0.07 + 0.85; 0.25 + 0.46; 0.61 +0.4; \textbf{0.67+0}) + 0.01 + 0.05 + 0.12 + 0.18 +0.03 + 0.4= 1.46$\\


\bigskip

$C[1; 6] = min(0 + 0.67 ; 0.21 + 1.41; 0.23 + 1.21; 0.34 + 0.85 ;\textbf{ 0.64+0.46}; 1.09 + 0.4 ; 1.18 + 0) + 0.21 + 0.01 + 0.05 + 0.12 + 0.18 + 0.03 + 0.4= 2.1$\\ 

\begin{center}
Matrice C : \\ \smallskip
\begin{tabular}{ |c| c| c| c |c |c |c |c | c|}
 \hline 
 
& 0 & 1 & 2 & 3 & 4  & 5 & 6 & 7  \\ 
 \hline 
 \hline
1 & 0 & 0.21 & 0.23 & 0.34  & 0.64 & 1.09 & 1.18 &2.1\\
2 &  & 0 & 0.01 & 0.07  & 0.25 & 0.61 & 0.67 &1.46\\
3 &  &  & 0 & 0.05  & 0.22 & 0.57 & 0.63 &1.41\\
4 &  &  &  & 0  & 0.12 & 0.42 & 0.48 &1.21\\
5 &  &  &  &   & 0 & 0.18 & 0.24 & 0.85\\
6 & &  &  &   &  & 0 & 0.03 & 0.46\\
7 &  & &  &   &  &  & 0 & 0.4\\
8 &  &  &  &  &  &  &  &0 \\

  \hline 
\end{tabular}
\end{center}

\begin{center}
Matrice  racine:\\
\smallskip
\begin{tabular}{ |c| c| c| c |c |c |c |c | c|}
 \hline 
 
& 0 & 1 & 2 & 3 & 4  & 5 & 6 & 7  \\ 
 \hline 
 \hline
1 & 0 & 1 & 1 & 1  & 1 & 4 & 4 &5\\
2 &  & 0 & 2 & 3  & 4 & 4/5 & 5 &7\\
3 &  &  & 0 & 3  & 4 & 5 & 5 &7\\
4 &  &  &  & 0  & 4 & 5 & 5 &7\\
5 &  &  &  &   & 0 & 5 & 5 & 7\\
6 & &  &  &   &  & 0 & 6 &7\\
7 &  & &  &   &  &  & 0 & 7\\
8 &  &  &  &  &  &  &  &0 \\

  \hline 
\end{tabular}
\end{center}
\underline{Arbre de recherche optimal} :

\begin{center}


\forestset{
    treenode/.style = {rectangle, draw, base=top},
    payoff/.style   = {align=center, base=top}
    }

\begin{forest}
  for tree={child anchor=north}
   [ 	[\texttt{$5$:$0,\!18$},no edge,treenode 
      		[\texttt{$1$:$0,\!21$},treenode,edge 
        		[\texttt{}, no edge]
      			[\texttt{$4$:$0,\!12$},treenode,edge 
      				[\texttt{$3$:$0,\!05$},treenode,edge 
      				  [\texttt{$2$:$0,\!01$}, treenode, edge]
                            [\texttt{}, no edge]
                            ]
                        [\texttt{}, no edge]
      			]
      		]
      		[\texttt{$7$:$0,\!4$},treenode,edge 
                    [\texttt{$6$:$0,\!03$}, treenode, edge]
                    [\texttt{}, no edge]
                    ]
    	]
    ]
  \end{forest}
\end{center}

\underline{Espérance du temps de recherche} : 2,1.
\newpage

\underline{Exercice 2} \textbf{(25 points)}


\begin{center}
\begin{tabular}{ |c| c c c c c c c c  c c |}
 \hline 
Tâche & 1 & 2 & 3 & 4  & 5 & 6 & 7 & 8 & 9 & 10 \\ 
 \hline 
Échéance & 3 & 2 & 2 & 3  & 4 & 5 & 1 & 5 & 6 & 8  \\
 Pénalité & 95 & 85 & 55 & 60  & 50 & 45 & 40 & 30 & 20 & 10 \\ 

  \hline 
\end{tabular}
\end{center}

\begin{center}
\begin{tabular}{ |c| c c c c c c c c  c c |}
 \hline 
 & & & 1 & & & & & & & \\ 
Tâche & 1 & 2 & 3 & 4  & 5 & 6 & 7 & 8 & 9 & 10 \\ 
 \hline 
Échéance & 3 & 2 & 2 & 3  & 4 & 5 & 1 & 5 & 6 & 8  \\
 \hline
 $N_{i}(F)$ & 0 & 0 & 1 & 1 & 1 & 1 & 1 & 1 & 1 & 1 \\
 
  \hline 
\end{tabular}
\end{center}

\begin{center}
\begin{tabular}{ |c| c c c c c c c c  c c |}
 \hline 
 & & 2 & 1 & & & & & & & \\ 
Tâche & 1 & 2 & 3 & 4  & 5 & 6 & 7 & 8 & 9 & 10 \\ 
 \hline 
Échéance & 3 & 2 & 2 & 3  & 4 & 5 & 1 & 5 & 6 & 8  \\
 \hline
 $N_{i}(F)$ & 0 & 1 & 2 & 2 & 2 & 2 & 2 & 2 & 2 & 2 \\
 
  \hline 
\end{tabular}
\end{center}

\begin{center}
\begin{tabular}{ |c| c c c c c c c c  c c |}
 \hline 
&  & 3 & & & & & & & & \\   
 & & 2 & 1 & & & & & & & \\ 
Tâche & 1 & 2 & 3 & 4  & 5 & 6 & 7 & 8 & 9 & 10 \\ 
 \hline 
Échéance & 3 & 2 & 2 & 3  & 4 & 5 & 1 & 5 & 6 & 8  \\
 \hline
 $N_{i}(F)$ & 0 & 2 & 3 & 3 & 3 & 3 & 3 & 3 & 3 & 3 \\
 
  \hline 
\end{tabular}
\end{center}

\begin{center}
\begin{tabular}{ |c| c c c c c c c c  c c |}
 \hline 
&  & 3 & {\color{magenta}4} & & & & & & & \\   
 & & 2 & 1 & & & & & & & \\ 
Tâche & 1 & 2 & 3 & 4  & 5 & 6 & 7 & 8 & 9 & 10 \\ 
 \hline 
Échéance & 3 & 2 & 2 & 3  & 4 & 5 & 1 & 5 & 6 & 8  \\
 \hline
 $N_{i}(F)$ & 0 & 2 & 3 & 3 & 3 & 3 & 3 & 3 & 3 & 3 \\
 
  \hline 
\end{tabular}
\end{center}

\begin{center}
\begin{tabular}{ |c| c c c c c c c c  c c |}
 \hline 
&  & 3 & {\color{magenta}4} & & & & & & & \\   
 & & 2 & 1 & 5 & & & & & & \\ 
Tâche & 1 & 2 & 3 & 4  & 5 & 6 & 7 & 8 & 9 & 10 \\ 
 \hline 
Échéance & 3 & 2 & 2 & 3  & 4 & 5 & 1 & 5 & 6 & 8  \\
 \hline
 $N_{i}(F)$ & 0 & 2 & 3 & 4 & 4 & 4 & 4 & 4 & 4 & 4 \\
 
  \hline 
\end{tabular}
\end{center}

\begin{center}
\begin{tabular}{ |c| c c c c c c c c  c c |}
 \hline 
&  & 3 & {\color{magenta}4} & & & & & & & \\   
 & & 2 & 1 & 5 & 6& & & & & \\ 
Tâche & 1 & 2 & 3 & 4  & 5 & 6 & 7 & 8 & 9 & 10 \\ 
 \hline 
Échéance & 3 & 2 & 2 & 3  & 4 & 5 & 1 & 5 & 6 & 8  \\
 \hline
 $N_{i}(F)$ & 0 & 2 & 3 & 4 & 5 & 5 & 5 & 5 & 5 & 5 \\
 
  \hline 
\end{tabular}
\end{center}

\begin{center}
\begin{tabular}{ |c| c c c c c c c c  c c |}
 \hline 
&  & 3 & {\color{magenta}4} & & & & & & & \\   
 & {\color{magenta}7}& 2 & 1 & 5 & 6& & & & & \\ 
Tâche & 1 & 2 & 3 & 4  & 5 & 6 & 7 & 8 & 9 & 10 \\ 
 \hline 
Échéance & 3 & 2 & 2 & 3  & 4 & 5 & 1 & 5 & 6 & 8  \\
 \hline
 $N_{i}(F)$ & 0 & 2 & 3 & 4 & 5 & 5 & 5 & 5 & 5 & 5 \\
 
  \hline 
\end{tabular}
\end{center}

\begin{center}
\begin{tabular}{ |c| c c c c c c c c  c c |}
 \hline 
&  & 3 & {\color{magenta}4} & & {\color{magenta}8} & & & & & \\   
 & {\color{magenta}7}& 2 & 1 & 5 & 6& & & & & \\ 
Tâche & 1 & 2 & 3 & 4  & 5 & 6 & 7 & 8 & 9 & 10 \\ 
 \hline 
Échéance & 3 & 2 & 2 & 3  & 4 & 5 & 1 & 5 & 6 & 8  \\
 \hline
 $N_{i}(F)$ & 0 & 2 & 3 & 4 & 5 & 5 & 5 & 5 & 5 & 5 \\
 
  \hline 
\end{tabular}
\end{center}

\begin{center}
\begin{tabular}{ |c| c c c c c c c c  c c |}
 \hline 
&  & 3 & {\color{magenta}4} & & {\color{magenta}8} & & & & & \\   
 & {\color{magenta}7}& 2 & 1 & 5 & 6& 9 & & & & \\ 
Tâche & 1 & 2 & 3 & 4  & 5 & 6 & 7 & 8 & 9 & 10 \\ 
 \hline 
Échéance & 3 & 2 & 2 & 3  & 4 & 5 & 1 & 5 & 6 & 8  \\
 \hline
 $N_{i}(F)$ & 0 & 2 & 3 & 4 & 5 & 6 & 6 & 6 & 6 & 6 \\
 
  \hline 
\end{tabular}
\end{center}

\begin{center}
\begin{tabular}{ |c| c c c c c c c c  c c |}
 \hline 
&  & 3 & {\color{magenta}4} & & {\color{magenta}8} & & & & & \\   
 & {\color{magenta}7}& 2 & 1 & 5 & 6& 9 & & 10 & & \\ 
Tâche & 1 & 2 & 3 & 4  & 5 & 6 & 7 & 8 & 9 & 10 \\ 
 \hline 
Échéance & 3 & 2 & 2 & 3  & 4 & 5 & 1 & 5 & 6 & 8  \\
 \hline
 $N_{i}(F)$ & 0 & 2 & 3 & 4 & 5 & 6 & 6 & 7 & 7 & 7 \\
 
  \hline 
\end{tabular}
\end{center}

\bigskip

\textbf{Ordonnancement optimal} : 3 2 1 5 6 9 10 {\color{magenta} 7} {\color{magenta} 4} {\color{magenta} 8}

\textbf{Pénalité} : 60 + 40 + 30 = 130

\newpage

\underline{Exercice 3} \textbf{(25 points)}
 
$$\left(\left( x_1 \rightarrow   x_2 \right) \land   x_1 \right)   \lor \lnot x_2 .$$
 \bigskip
 
 \underline{Étape 1} : 
 
 \bigskip

 \begin{center}
\forestset{
    treenode/.style = {circle, draw, base=top},
    payoff/.style   = {align=center, base=top}
    }

 \begin{forest}
  for tree={child anchor=north,minimum size=8 mm}
   [ 	[$\lor$, edge label={node[midway, right]{$y_1$}},treenode  
      		[$\land$,treenode,edge label={node[below = -0.15cm,left]{$y_2$}}
        		[$\to$,treenode,edge label={node[below = -0.15cm,left]{$y_3$}}
					[$ x_1$]
        			[$ x_2$]        		
        		]
        		[$ x_1$]
      		]
                [$\lnot x_2$]
    	]
    ]
  \end{forest}
\end{center}

\begin{center}
    $\phi_1^{'} = y_1$

    $\phi_2^{'}= y_1 \leftrightarrow (y_2 \lor \lnot x_2)$

    $\phi_3^{'}= y_2 \leftrightarrow (y_3 \land x_1)$

    $\phi_4^{'}= y_3 \leftrightarrow (x_1 \rightarrow x_2)$
\end{center}

\smallskip

$\phi^{'} = y_1 \land (y_1 \leftrightarrow (y_2 \lor \lnot x_2)) \land (2 \leftrightarrow (y_3 \land x_1)) \land (y_3 \leftrightarrow (x_1 \rightarrow x_2))$

\smallskip

$\phi^{'} = \phi_1^{'} \land \phi_2^{'}= y_1 \land \phi_3^{'}= y_1 \land \phi_4^{'}= y_1$

\bigskip

\underline{Étape 2}: 

\bigskip

\begin{center}
\begin{tabular}{ |c|c|c|c|}
 \hline 
 $y_2$ & $y_3$ & $x_1$ & $y_1 \leftrightarrow (y_2 \lor \lnot x_2)$ \\   
 \hline 
0 & 0 & 0 & 0 \\
  \hline 
0 & 0 & 1 & 1 \\
  \hline 
0 & 1 & 0 & 0 \\
  \hline 
0 & 1 & 1 & 0 \\
  \hline 
1 & 0 & 0 & 1 \\
  \hline 
1 & 0 & 1 & 0 \\
  \hline 
1 & 1 & 0 & 1 \\
  \hline 
1 & 1 & 1 & 1 \\
  \hline 
\end{tabular}
\end{center}

\smallskip

\begin{center}
    $\lnot \phi_2^{''} = (\lnot y_1 \land \lnot y_2 \land \lnot x_2) \lor (\lnot y_1 \land y_2 \land \lnot x_2) \lor (\lnot y_1 \land y_2 \land x_2) \lor (y_1 \land \lnot y_2 \land \lnot x_2) $ \\
    $\phi_2^{''} = ( y_1 \lor y_2 \lor x_2) \land ( y_1 \lor \lnot y_2 \lor x_2) \land ( y_1 \lor \lnot y_2 \lor \lnot x_2) \land (\lnot y_1 \lor y_2 \lor x_2) $
\end{center}

\bigskip

\begin{center}
\begin{tabular}{ |c|c|c|c|}
 \hline 
 $y_1$ & $y_2$ & $x_1$ & $y_2 \leftrightarrow (y_3 \land x_1)$ \\   
 \hline 
0 & 0 & 0 & 1 \\
  \hline 
0 & 0 & 1 & 1 \\
  \hline 
0 & 1 & 0 & 1 \\
  \hline 
0 & 1 & 1 & 0 \\
  \hline 
1 & 0 & 0 & 0 \\
  \hline 
1 & 0 & 1 & 0 \\
  \hline 
1 & 1 & 0 & 0 \\
  \hline 
1 & 1 & 1 & 1 \\
  \hline 
\end{tabular}
\end{center}

\smallskip

\begin{center}
    $\lnot \phi_3^{''} = (\lnot y_2 \land \lnot y_3 \land  x_1) \lor (y_2 \land \lnot y_3 \land \lnot x_1) \lor (y_2 \land \lnot y_3 \land x_1) \lor (y_2 \land y_3 \land \lnot x_1) $ \\
    $\phi_3^{''} = (y_2 \lor y_3 \lor \lnot x_1) \land (\lnot y_2 \lor y_3 \lor x_1) \land (\lnot y_2 \lor y_3 \lor \lnot x_1) \land (\lnot y_2 \lor \lnot y_3 \lor x_1) $
\end{center}

\bigskip

\begin{center}
\begin{tabular}{ |c|c|c|c|}
 \hline 
 $y_3$ & $x_1$ & $x_2$ & $\phi_4^{'}= y_3 \leftrightarrow (x_1 \rightarrow x_2)$ \\   
 \hline 
0 & 0 & 0 & 0 \\
  \hline 
0 & 0 & 1 & 0 \\
  \hline 
0 & 1 & 0 & 1 \\
  \hline 
0 & 1 & 1 & 0 \\
  \hline 
1 & 0 & 0 & 1 \\
  \hline 
1 & 0 & 1 & 1 \\
  \hline 
1 & 1 & 0 & 0 \\
  \hline 
1 & 1 & 1 & 1 \\
  \hline 
\end{tabular}
\end{center}

\smallskip

\begin{center}
    $\lnot \phi_4^{''} = (\lnot y_3 \land x_1 \land \lnot x_1) \lor (\lnot y_3 \land \lnot x_1 \land x_2) \lor (\lnot y_3 \land x_1 \land x_2) \lor (y_3 \land x_1 \land \lnot x_2) $ \\
    $\phi_4^{''} = (y_3 \lor \lnot x_1 \lor x_1) \land (y_3 \lor x_1 \lor \lnot x_2) \land (y_3 \lor \lnot x_1 \lor \lnot x_2) \land (\lnot y_3 \lor \lnot x_1 \lor x_2) $
\end{center}

\bigskip

\underline{Etape 3}:

\bigskip

\begin{center}
    $\phi_1 = \phi_1 = (y_1 \lor z_1 \lor z_2) \land (y_1 \lor \lnot z_1 \lor z_2) \land (y_1 \lor z_1 \lor \lnot z_2) \land (y_1 \lor \lnot z_1 \lor \lnot z_2)$\\
    \smallskip
    $\phi_2^{'''} = \phi_2^{''}$\\
    \smallskip
    $\phi_3^{'''} = \phi_3^{''}$\\
    \smallskip
    $\phi_4^{'''} = \phi_4^{''}$
\end{center}


\newpage

\underline{Exercice 4} \textbf{(25 points)}

\begin{enumerate}[label=\Roman*.]

\item 
\underline{Ligne 1} :
\smallskip

$SM[1][2]:$\\
pour $k=0$, $SM[1][2]= [1][1] + S[0][2] = 15+0 = 15$ \\
pour $k=1$, $SM[1][2]= [0][1] + S[1][2] = 0+ 20 = \underline{20}$ \\
\smallskip

$SM[1][3]:$\\
pour $k=0$, $SM[1][3]= [1][2] + S[0][3] = 20 +0 = \underline{20}$ \\
pour $k=1$, $SM[1][3]= [0][2] + S[1][3] = 0+ 18 = 18$ \\
\smallskip

$SM[1][4]:$\\
pour $k=0$, $SM[1][4]= [1][3] + S[0][3] = 20+0 = 15$ \\
pour $k=1$, $SM[1][4]= [0][3] + S[1][3] = 0+ 25 = \underline{25}$ \\
\smallskip

 \underline{Ligne 2} : 
 
 \smallskip

 $SM[2][2]:$\\
pour $k=0$, $SM[2][2]= [2][1] + S[0][2] = 30+0 = 30$ \\
pour $k=1$, $SM[2][2]= [1][1] + S[1][2] = 15+ 20 = 35$ \\
pour $k=2$, $SM[2][2]= [0][1] + S[2][2] = 0+ 40 = \underline{40}$ \\
\smallskip

$SM[2][3]:$\\
pour $k=0$, $SM[2][3]= [2][2] + S[0][3] = 40 +0 = \underline{40}$ \\
pour $k=1$, $SM[2][3]= [1][2] + S[1][3] = 20+ 18 = 38$ \\
pour $k=2$, $SM[2][3]= [0][2] + S[2][3] = 0+ 38 = 38$ \\
\smallskip

$SM[2][4]:$\\
pour $k=0$, $SM[2][4]= [2][3] + S[0][4] = 40+0 = 40$ \\
pour $k=1$, $SM[2][4]= [1][3] + S[1][4] = 20+ 25 = \underline{45}$ \\
pour $k=2$, $SM[2][4]= [0][3] + S[2][4] = 0+ 35 = 35$ \\
\smallskip

 \underline{Ligne 3} : 
 
 \smallskip

 $SM[3][2]:$\\
pour $k=0$, $SM[3][2]= [3][1] + S[0][2] = 45+0 = 45$ \\
pour $k=1$, $SM[3][2]= [2][1] + S[1][2] = 30+ 20 = 50$ \\
pour $k=2$, $SM[3][2]= [1][1] + S[2][2] = 15+ 40 = \underline{55}$ \\
pour $k=3$, $SM[3][2]= [0][1] + S[3][2] = 0+ 54 = 54$ \\
\smallskip

$SM[3][3]:$\\
pour $k=0$, $SM[3][3]= [3][2] + S[0][3] = 55 +0 = \underline{55}$ \\
pour $k=1$, $SM[3][3]= [2][2] + S[1][3] = 40+ 18 = 58$ \\
pour $k=2$, $SM[3][3]= [1][2] + S[2][3] = 20+ 38 = 58$ \\
pour $k=3$, $SM[3][3]= [0][2] + S[3][3] = 0+ 59 = \underline{59}$ \\
\smallskip

$SM[3][4]:$\\
pour $k=0$, $SM[3][4]= [3][3] + S[0][4] = 59+0 = 59$ \\
pour $k=1$, $SM[3][4]= [2][3] + S[1][4] = 40+ 25 = \underline{65}$ \\
pour $k=2$, $SM[3][4]= [1][3] + S[2][4] = 20+ 35 = 55$ \\
pour $k=3$, $SM[3][4]= [0][3] + S[3][4] = 0+ 47 = 47$ \\
\smallskip

 \underline{Ligne 4} : 
 
 \smallskip

 $SM[4][2]:$\\
pour $k=0$, $SM[4][2]= [4][1] + S[0][2] = 60+0 = 60$ \\
pour $k=1$, $SM[4][2]= [3][1] + S[1][2] = 45+ 20 = 65$ \\
pour $k=2$, $SM[4][2]= [2][1] + S[2][2] = 30+ 40 = \underline{70}$ \\
pour $k=3$, $SM[4][2]= [1][1] + S[3][2] = 15+ 54 = 69$ \\
pour $k=4$, $SM[4][2]= [0][1] + S[4][2] = 0+ 65 = 65$ \\
\smallskip

$SM[4][3]:$\\
pour $k=0$, $SM[4][3]= [4][2] + S[0][3] = 70 +0 = 70$ \\
pour $k=1$, $SM[4][3]= [3][2] + S[1][3] = 55+ 18 = 73$ \\
pour $k=2$, $SM[4][3]= [2][2] + S[2][3] = 40+ 38 = 78$ \\
pour $k=3$, $SM[4][3]= [1][2] + S[3][3] = 20+ 59 = \underline{79}$ \\
pour $k=4$, $SM[4][3]= [0][2] + S[4][3] = 0+ 68 = 68$ \\
\smallskip

$SM[4][4]:$\\
pour $k=0$, $SM[4][4]= [4][3] + S[0][4] = 79+0 = 59$ \\
pour $k=1$, $SM[4][4]= [3][3] + S[1][4] = 59+ 25 = \underline{84}$ \\
pour $k=2$, $SM[4][4]= [2][3] + S[2][4] = 40+ 35 = 75$ \\
pour $k=3$, $SM[4][4]= [1][3] + S[3][4] = 20+ 47 = 67$ \\
pour $k=4$, $SM[4][4]= [0][3] + S[4][4] = 0+ 69 = 69$ \\
\smallskip

 \underline{Ligne 5} : 
 
 \smallskip

 $SM[5][2]:$\\
pour $k=0$, $SM[5][2]= [5][1] + S[0][2] = 75+0 = 75$ \\
pour $k=1$, $SM[5][2]= [4][1] + S[1][2] = 60+ 20 = 80$ \\
pour $k=2$, $SM[5][2]= [3][1] + S[2][2] = 45+ 40 = \underline{85}$ \\
pour $k=3$, $SM[5][2]= [2][1] + S[3][2] = 30+ 54 = 84$ \\
pour $k=4$, $SM[5][2]= [1][1] + S[4][2] = 15+ 55 = 80$ \\
pour $k=5$, $SM[5][2]= [0][1] + S[5][2] = 0+ 70 = 70$ \\
\smallskip

$SM[5][3]:$\\
pour $k=0$, $SM[5][3]= [5][2] + S[0][3] = 85 +0 = 85$ \\
pour $k=1$, $SM[5][3]= [4][2] + S[1][3] = 70+ 18 = 88$ \\
pour $k=2$, $SM[5][3]= [3][2] + S[2][3] = 55+ 38 = 83$ \\
pour $k=3$, $SM[5][3]= [2][2] + S[3][3] = 40+ 59 = \underline{99}$ \\
pour $k=4$, $SM[5][3]= [1][2] + S[4][3] = 20+ 68 = 88$ \\
pour $k=5$, $SM[5][3]= [0][2] + S[5][3] = 0+ 78 = 78$ \\
\smallskip

$SM[5][4]:$\\
pour $k=0$, $SM[5][4]= [5][3] + S[0][4] = 99+0 = 99$ \\
pour $k=1$, $SM[5][4]= [4][3] + S[1][4] = 79+ 25 = \underline{104}$ \\
pour $k=2$, $SM[5][4]= [3][3] + S[2][4] = 59+ 35 = 94$ \\
pour $k=3$, $SM[5][4]= [2][3] + S[3][4] = 40+ 47 = 87$ \\
pour $k=4$, $SM[5][4]= [1][3] + S[4][4] = 20+ 69 = 89$ \\
pour $k=5$, $SM[5][4]= [0][3] + S[5][4] = 0+ 80 = 80$ \\
\smallskip

\bigskip

$$SM=\left( \begin{matrix}
0& 0& 0 & 0   \\
15& 20& 20 & 25   \\
30& 40& 40 & 45   \\
45& 55& 59 & 65   \\
60& 70& 79 & 84   \\
75& 85& 99 & 104   \\
\end{matrix}  \right) \,\,\,\,  \,\,\,\,  \,\,\,\,  \,\,\,\, NH=\left( \begin{matrix}
0& 0& 0 & 0   \\
1& 1& 0 & 1   \\
2& 2& 0 & 1   \\
3& 2& 3 & 1   \\
4& 2& 3 & 1   \\
5& 2& 3 & 1   \\
\end{matrix}  \right).  $$

\item Le revenu maximal est de 104\$ pour 1 heure de travail dans l'entreprise 4, 3 heures dans l'entreprise 3 et 2 dans l'entreprise 2.
Selon la matrice $NH$, on doit regarder les dernières non nulles de chaque colonne. Cela correspond au nombre d'heures que l'étudiant doit travailler dans chaque entreprise pour un revenu maximal.
\end{enumerate}




\end{document}





